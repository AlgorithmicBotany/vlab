\chapter {Texturing}

Textures are used to modify the appearance of an object through the
use of procedural functions.  A texture may modify any surface characteristic,
such as diffuse color, reflectivity, or transparency, or it may 
modify the surface normal (``bump mapping'') in order to give the
appearance of a rough surface.

Any number of textures may be associated with an object.  If more than
one texture is specified, they are applied in the order given.  This allows
one to compose texturing functions and create, for example
a tiled marble ground plane using the {\em checker} and {\em marble}
textures.

Textures are associated with objects by following the object specification
by a number of lines of the form:

\begin{center}
{\tt texture} {\em name} $<${\em Texturing Arguments}$>$ [{\em Transformations}]
\end{center}

Transformations may be applied to the texture in order to, for example,
shrink or grow feature size, change the orientation of features, and
change the position of features.

Several of the texturing functions take the name of a colormap as an
argument.  A colormap is 256-line ASCII file, with each line containing
three space-separated values ranging from 0 to 255.  Each line gives
the red, green, and blue values for a single entry in the colormap.

\section {Texturing Functions}

\begin{defkey}{blotch}{{\em BlendFactor surface}}
Produces a mildly interesting blotchy-looking surface.
{\em BlendFactor} is used to control the interpolation between
the default surface characteristics and the characteristics of
the given surface.  A value of 0 results in a roughly 50-50 mix
of the two surfaces.  Higher values result in a great portion of
the default surface characteristics.
\end{defkey}

\begin{defkey}{bump}{{\em scale}}
Apply a random bump map.  The point of intersection is passed to
{\em DNoise()}.
The returned normalized vector is weighted by {\em scale}
and the result is added to the normal vector at the point of intersection.
\end{defkey}
Using an image texture applied to the ``bump'' component offers a more
direct way to control the modification of surface normals (see below).

\begin{defkey}{checker}{$<${\em Surface}$>$}
Applies a 3D checkerboard texture.  Every point that falls within an
``even'' unit cube will be assigned the characteristics of the named surface
applied to it, while points that fall within ``odd'' cubes will have
its usual surface characteristics.  Be wary of strange effects due
to roundoff error that occur when a planar checkered surface lies
in a plane of constant integral value (e.g., $z=0$) in texture space.
In such cases,
simply translate the texture to ensure that the planar surface is not
coincident with an integral plane in texture space
(e.g., {\tt translate 0 0 0.1}).
\end{defkey}

\begin{defkey}{cloud}{{\em scale H $\lambda$ octaves cthresh lthresh tscale}}
	This texture is a variant on Geoff Gardner's ellipsoid-texturing
	algorithm.  It should be applied to unit spheres centered
	at the origin.  These spheres may, of course, be transformed
	at will to form the appropriately-shaped cloud or tree.

	A sample of normalized {\em fBm} (see the {\em fbm} texture) is
	generated
	at the point of intersection.  This sample is used to
	modulate the surface transparency.  The final transparency
	if a function of the sample value, the
	the proximity of the point of intersection to the edge of
	the sphere (as seen from the ray origin), and three parameters
	to control the overall ``density.''  The proximity of the point
	to the sphere edge is determined by evaluating a {\em limb} function,
	which varies from 0 on the limb to 1 at the center of the sphere.
\[
transp = 1. - \frac{fBm - cthresh - (lthresh - cthresh)(1 - limb)}{tscale}
\]
\end{defkey}

\begin{defkey}{fbm}{{\em offset scale H $\lambda$ octaves thresh}
[{\em colormap}]}
Generate a sample of discretized fractional Brownian motion (fBm) and
uses it to scale the diffuse and ambient component of an object's surface.
{\em Scale} is used to scale the value
returned by the fBm function.  {\em Offset} allows one to control the minimum
value of the fBm function.  {\em H} is the {\em Holder exponent}
used in the fBm function (a value of 0.5 works well).  $\lambda$ is
used to control {\em lacunarity}, and specifies the the frequency
difference between successive samples of the fBm basis function (a
value of 2.0 will suffice).  {\em Octaves} specifies the number of
octaves (samples) to take of the fBm basis function (in this case, Noise()).
Between five and seven octaves usually works well.  {\em Thresh} is used
to specify a lower bound onthe output of the fBm function.  Any
value lower than {\em thresh} is set to zero.

If a {\em colormap} is named, a 256-entry colormap is read from the named
file, and the sample of fBm is scaled by 255 and is used as an index into
the colormap.  The resulting colormap entry
is used to scale the ambient and diffuse components of the
object's surface.
\end{defkey}

\begin{defkey}{fbmbump}{{\em offset scale H $\lambda$ octaves}}
Similar to the {\em fbm} texture.  Rather than modifying the color of
a surface, this texture acts as a bump map.
\end{defkey}

\begin{defkey}{gloss}{{\em glossiness}}
Gives reflective surfaces a glossy appearance. This texture perturbs
the object's surface normal such that the normal ``samples'' a cone of
unit height with radius $1. - glossiness$.  A value of 1 results
in perfect mirror-like reflections, while a value of 0 results
in extremely fuzzy reflections.  For best results, jittered sampling
should be used to render scenes that make use of this texture.
\end{defkey}

\begin{defkey}{marble}{[{\em colormap}]}
Gives a surface a marble-like appearance.  The texture is implemented as
roughly parallel alternating veins of marble, each of which is
separated by 1/7 of a unit and runs perpendicular to the Z axis.
If a colormap is named, the surface's ambient and diffuse colors
will be scaled using the RGB values in the colormap.  If no colormap is
given, the diffuse and ambient components are simply scaled by the
value of the marble function.  One may transform the texture to
control the density and orientation of the marble veins.
\end{defkey}

\begin{defkey}{sky}{{\em scale H $\lambda$ octaves cthresh ltresh}}
	Similar to the {\em fbm} texture.  Rather than modifying the
	color of a surface, this texture modulates its transparency.
	{\em cthresh} is the value of the {\em fBm} function above
	which the surface is totally opaque.  Below {\em lthresh},
	the surface is totally transparent.
\end{defkey}

\begin{defkey}{stripe}{$<${\em Surface}$>$ {\em size bump} $<$Mapping$>$}
	Apply a ``raised'' stripe pattern to the surface.
	The surface properties used to color the stripe are those
	of the given surface.  The width of the stripe, as compared
	to the unit interval, is given by {\em size}.  The magnitude
	of {\em bump} controls the extent to which the bump appears
	to be displaced from the rest of the surface.  If negative,
	the stripe will appear to
	sink into the surface; if positive, it will appear to stand
	out of the surface.
\end{defkey}
Mapping functions are described below.

\begin{defkey}{wood}{}
Gives a surface a wood-like appearance.  The feature size of this texture
is approximately $0.01$ of a unit, making it often necessary to
scale the texture in order to achieve the desired appearance.
\end{defkey}

\section {Image Texturing}

{\Rayshade} also supports an {\tt image} texture.  This texture
allows you to use images to modify the characteristics of a surface.  You
can use three-channel images to modify the any or all of
the ambient, diffuse, and specular colors of a surface.
If you are using the Utah Raster Toolkit,
you can also use single-channel images to modify surface reflectance,
transparency, and the specular exponent.  You can also use a single-channel
image to apply a bump map to a surface.

In all but the bump-mapping case,
a component is modified by multiplying the given value by the value
computed by the texturing function.  When using the Utah Raster Toolkit,
surface characteristics are modified in proportion to the value of
the {\em alpha} channel in the image.  If there is no {\em alpha} channel,
or you are not using the Utah Raster Toolkit, {\em alpha} is assumed to be
everywhere
equal to $1$.

\begin{defkey}{component}{$<${\em Component}$>$}
	The named component will be modified.
\end{defkey}
Possible surface components are:
{\tt ambient} (modify ambient color),
{\tt diffuse} (modify diffuse color),
{\tt specular} (modify specular color),
{\tt specpow}, (modify specular exponent),
{\tt reflect},	(modify reflectivity),
{\tt transp} (modify transparency),
{\tt bump}, (modify surface normal).
The {\tt specpow}, {\tt reflect}, {\tt transp}, and {\tt bump}
components require the use of a single-channel image.

\begin{defkey}{range}{{\em high low}}
	Specify the range of values to which the values in the
	image should be mapped.  An value of $1$ will
	be mapped {\em high}, $0$ to {\em low}.  Intermediate
	values will be linearly interpolated.
\end{defkey}

\begin{defkey}{smooth}{}
	When given, pixel averaging will be performed in order
	to smooth the sampled image.  If not specified, no averaging
	will occur.
\end{defkey}

\begin{defkey}{textsurf}{$<${\em Surface Specification}$>$}
	For use when modifying surface colors, this keyword specifies
	that the given surface should be used as the base
	to be modified when the {\em alpha} value in the image
	is non-zero.  When {\em alpha} is zero, the object's
	unmodified default surface characteristics are retained.
\end{defkey}
The usual behavior is for the object's default surface properties to
be used.

\begin{defkey}{tile}{{\em un vn}}
	Specify how the image should be tiled (repeated) along the
	$u$ and $v$ axes.
	If positive, the value of {\em un} gives the number of
	times the image should be repeated along the $u$ axis, starting
	from the origin of the texture, and positive {\em vn} gives the
	number of times it
	should be repeated along the $v$ axis.  If either value is zero,
	the image is repeated infinitely along the appropriate axis.
\end{defkey}
Tiling is usually only a concern when planar mapping is being used,
though it may also be used if image textures are being scaled.  By default
{\em un} and {\em vn} are both zero.
 

A mapping function may also be associated with an image texture.

\section {Mapping Functions}

Mapping functions are used to apply two-dimensional textures to
surfaces.  Each mapping functions defines a different method of transforming
a three dimensional point of intersection to a two dimensional $u-v$ pair
termed texturing coordinates.
Typically, the arguments to a mapping method define a center of
a projection and two non-parallel axes that define a local coordinate system.

The default mapping method is termed $u-v$ mapping or {\em inverse mapping}.
Normally, there is a different inverse mapping method for each primitive type
(see chapter 5).
When inverse mapping is used, the point of intersection is passed to
the $uv$ method for the primitive that was hit.

\begin{defkey}{map}{{\tt uv}}
	Use the $uv$ (inverse mapping) method associated with the
	object that was intersected in order to map from 3D to determine
	texturing coordinates.
\end{defkey}
The inverse mapping method for each primitive is described in Chapter 5.

\begin{defkey}{map}{{\tt planar} [\evec{origin} \evec{vaxis} \evec{uaxis}]}
	Use a planar mapping method. The 2D texture is transformed
	so that its $u$ axis is given by \evec{uaxis} and its $v$
	axis by $vaxis$.  The texture is projected along the vector
	defined by the cross product of the $u$ and $v$ axes, with
	the (0,0) in texture space mapped to \evec{origin}.
\end{defkey}

\begin{defkey}{map}{{\tt cylindrical} [\evec{origin} \evec{vaxis} \evec{uaxis}]}
	Use a cylindrical mapping method.  The point of intersection
	is projected onto an imaginary cylinder, and the location
	of the projected point is used to determine the texture coordinates.
	If given, \evec{origin} and
	\evec{vaxis} define the cylinder's axis, and \evec{uaxis} defines
	where $u=0$ is located.
\end{defkey}
See the description of the inverse mapping method for the 
cylinder in Chapter 5.  By default, the point of intersection is
projected onto a cylinder that runs through the origin along the $z$
axis, with \evec{uaxis} equal to the $x$ axis.

\begin{defkey}{map}{{\tt spherical} [\evec{origin} \evec{vaxis} \evec{uaxis}]}
	Use a spherical mapping method.  The intersection point is
	projected onto an imaginary sphere, and the location of the
	projected point	 is used to determine the texturing coordinates
	in a manner identical to that used in the inverse mapping method
	for the sphere primitive.
	If given, the center of
	the projection is \evec{origin}, \evec{vaxis} defines
	the sphere axis, and the point where the
	non-parallel \evec{uaxis} intersects the sphere
	defines where $u=0$ is located.
\end{defkey}
By default, a spherical mapping projects points towards the origin,
with \evec{vaxis} defined to be the $z$ axis and
\evec{uaxis} defined to be the $x$ axis.
