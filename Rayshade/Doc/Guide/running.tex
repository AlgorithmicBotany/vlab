\chapter{Running Rayshade}

{\Rayshade}
can take anywhere from seconds to weeks to render an image.  The exact
time required is a function of the speed of the
machine(s) on which
you're working, the complexity of the scene, and how ``good'' you want
the final image to be.  \footnote{Appendix D describes some simple
ways to accelerate the rendering process.}

Creating a finished ray-traced image is an iterative process.  Usually, many
test renderings are made at low resolution and with
non-essential features turned off.  After each test image is created,
surface definitions might be modified, the eye or look positions may be
slightly changed, or the intensity of a light source changed.

This chapter describes the basic operation of {\rayshade} and some of
the options that control that operation.
Setting these options properly can greatly reduce
rendering time, improve the quality of your images, and make you a better
person.

{\Rayshade} usually works as a filter, reading a description from
the standard input and writing
an image file to the standard output.  As it is working, {\rayshade} reports
on the progress of the rendering by writing messages to the standard
error.

\section{The Input File}

The scene description read by {\rayshade} consists of a number of
keywords, each followed by a set of arguments.  These keywords can
be thought of as commands that direct {\rayshade} to do various things,
such as create objects, set the eye's position, and change an object's
appearance.

Many of the keywords have related command-line options for turning
on special features and setting values.
These options override the values given in the input file,
and are explained in detail in Appendix A.

{\Rayshade} is ``case sensitive,'' which means that typing
{\tt SPHERE} or {\tt Sphere} instead of {\tt sphere} won't work.
{\Rayshade} keywords are all lower-case.  Many people choose to
capitalize the first letter of names that they give to objects
or surfaces in order to make then ``stand out'' in an input file.

Keywords, numbers and strings in the input file are separated by spaces,
tabs, or new lines (carriage returns).  These ``whitespace'' characters
are handled identically by {\rayshade}, which means that you can separate
keywords from keywords, key words from arguments, and arguments from
arguments using any combination of whitespace characters that you choose.

\begin{table}
\centering
\begin{tabular}{|c|c|l|}\hline\hline
Operator &    Use    & Operation \\ \hline \hline
\verb+^+ &  \verb!a ^ b!  & Exponentiation \\  \hline
\verb+-+ &  \verb! -a  !  & Negation  \\   \hline
\verb+*+ &  \verb!a * b!  & Multiplication  \\  \hline
\verb+/+ &  \verb!a / b!  & Division     \\  \hline
\verb+%+ &  \verb!a % b!  & Remainder    \\  \hline
\verb-+- &  \verb!a + b!  & Addition     \\ \hline
\verb+-+ &  \verb!a - b!  & Subtraction  \\ \hline \hline
\end{tabular}
\caption{Operators, in order of decreasing precedence.}
\label{tab:operators}
\end{table}

Numbers may be entered directly as reals or as parenthesized expressions.
Reals may be written in exponential notation if you wish, and
integers may be written with or without a trailing decimal point.
When an integer value is called for and a real is given,
the real value is truncated and the resulting integer is used.
Table \ref{tab:operators} lists the available operators
available for use in expressions, in order of decreasing precedence.

The upshot of all this is that the strings
{\tt 42}, {\tt 42.},
{\tt (20 + 22)}, and {\tt (7\verb!^!2 - 7)} mean the same thing
to {\rayshade}.

Variables may also be defined and used in expressions.  Several
built-in functions are also provided.  See Appendix B for further details.

{\Rayshade} will automatically run the input it receives
through the C preprocessor if it is available.  This
allows you to use C preprocessor directives, such as {\tt \#include},
{\tt \#define}, and {\tt \#ifdef} when designing your input files.

Comments may be included in {\rayshade} input files by enclosing
text between the strings \verb!/*! and \verb!*/!, as in the C
programming
language.

\section{Images}

The end result of running {\rayshade} is an image file.  Depending upon
how it was installed, {\rayshade} writes images in either the Utah Raster
{\em RLE} format or a generic but easily-manipulated {\em mtv} format
used by Mark VandeWettering in his {\em mtv} ray tracer.  The {\em mtv}
format consists of a header giving the resolution of the image followed
by interleaved red-green-blue values for each pixel.  The {\em RLE} format
supports an arbitrary number of color channels,
an alpha channel, comments, a history field, and the ability
to treat images as windows into a larger image.  As a result of this
flexibility, a number of {\rayshade}'s features
are not
supported if the {\em mtv} format is being used. You are thus strongly
encouraged to obtain a copy of the Utah Raster Toolkit.

If the {\em mtv}
format is used, the image will (and must) consist of three eight-bit
color channels.
If the {\em RLE} format is used, the image file consists of three
eight-bit color channels plus an eight-bit alpha channel.  {\Rayshade}
also documents in the {\em RLE} header the command line and {\em gamma}
value used in creating the image.

If more than one frame is rendered, the resulting images are appended
in turn
to the image file.  The various utilities provided by the
Utah Raster Toolkit can be used to manipulate the resulting ``movie''
files.
If the Toolkit is not being used, you will probably need to write
utility programs to handle the decidedly non-standard multi-image {\em mtv}
format files.

By default, {\rayshade} writes the computed image to the standard output.
The image file may be written to a file instead by specifying a file name
in the input stream.

\begin{defkey}{outfile}{{\em filename}}
	Write the computed image to the named file.
\end{defkey}
The output file name may also be specified on the command line by
using the {\tt -O} option.

The size of the output image is measured in pixels.  The {\em x} size
is the number of pixels left-to-right, while the {\em y} size is
the number of pixels bottom-to-top.

\begin{defkey}{screen}{{\em xsize ysize}}
	Use a screen {\em xsize} pixels wide by {\em ysize} pixels high.
\end{defkey}
The screen size may also be set on the command line through
the {\tt -R} option.
The default
screen size is 512 by 512 pixels.

When rendering an image, it is often advantageous to split the image
into a number of disjoint windows, each of which is rendered
on a different machine.  One then combines the images corresponding to
the windows into a final image.

\begin{defkey}{window}{{\em minx maxx miny maxy}}
	Render the image in the given window.
\end{defkey}
The window must be properly
contained within the screen, i.e., {\em minx} and {\em miny} must
be greater than or equal to zero, while {\em maxx} and
{\em maxy} must be less than {\em xsize} and {\em ysize}, respectively.
The Utah Raster tool
{\em rlecomp\/} is useful for reconstructing the full image from
sub-images.
By default, the window
is equivalent to the entire screen.

It is also convenient to be able to render a small portion
of the window by specifying a subregion using normalized coordinates.

\begin{defkey}{crop}{{\em left right bottom top}}
	Crop the rendering window.
\end{defkey}
The rendering window is cropped by rendering the screen
area that falls within $minx + left(maxx - minx)$ and
$minx + right(maxx - minx)$ in the $X$ direction, and similarly
for the $Y$ direction.
{\em Left} and {\em bottom} must be greater than or equal to zero.
{\em Right} and {\em top} must be less than or equal to one.
If {\em left} is greater than {\em right}, the two values are
swapped, and similarly for {\em bottom} and {\em top}.

{\em Gamma correction} may also be applied to the three output color
channels.  See Appendix A for more details.

\section{Statistics Reporting}

As it is working, {\rayshade} informs you of its progress by writing
messages to a ``report file''.  By default, the report file is the
standard error.  The report itself consists of a number of
progress report lines consisting of the number of eye rays traced,
the total elapsed time, and the elapsed time since the last progress report.
The end of the report contains detailed statistics about intersection
tests performed, the number of rays traced, and the like.

\begin{defkey}{report}{[{\tt verbose}] [{\tt quiet}] [{\em freq}] [{\em file}]}
	Specify the kind of
	information included in the report and to which
	file it should be written.
	If {\tt verbose} is specified, the
	report will also include a listing of the options selected,
	the 
	bounding volumes of each aggregate,
	and the total number of primitives in each aggregate.
	{\tt quiet} causes warning
	messages to be suppressed.  {\em freq} specifies the frequency,
	in scanlines rendered, that progress reports are made.
	If given, {\em file} names a file to which the report
	will be written.
\end{defkey}
By default, a non-verbose, non-quiet report is 
written to the standard error once every 10 lines.
The {\tt -V} option may also be used to direct the report to a named file.

\section{Antialiasing}

Given a screen of a fixed size, creating an image is accomplished by
sampling each pixel one or more times in order to determine what can
be seen ``through'' that pixel by the camera.  A pixel thus covers
a square area of the image plane, not just a single point.

If a pixel is not sampled at the proper rate, aliasing will result.
Aliasing usually appears as ``jaggies'' or ``stair steps'' in the image.
In order to reduce these and other artifacts, {\rayshade} provides
an adaptive jittered antialiasing scheme that attempts to detect where
increased sampling rates are needed.
In jittered sampling, the location at which a sample is taken is
perturbed by a random amount.  This perturbation reduces aliasing
but adds noise to the image.  Appendix B describes how jittered
time sampling is implemented in {\rayshade}.

The adaptive sampling scheme implemented in {\rayshade} begins
by sampling each pixel on the current scanline once.
For each pixel on the scanline, the contrast between it and its
four immediate neighbors is computed.  If this contrast is greater
than a user-specified maximum in any color channel,
the pixel and its
neighbors are all supersampled by firing an additional
{$numsamples^2 -1$} rays through those pixels that have not already been
supersampled.  This process is repeated for the current scanline
until a pass is made without any
pixel being supersampled.

\begin{defkey}{contrast}{{\em redcont greencont bluecont}}
	Set the maximum allowed contrast between four color
	samples when adaptive supersampling is used.
	The contrast test is applied to each color
	channel separately.
\end{defkey}
The default maximum contrast values for the red, green, and blue
channels are 0.25, 0.2, and 0.4, respectively.

\begin{defkey}{sample}{{\em n} [{\tt nojitter}]}
	Use $n^{2} $ samples when performing jittered
	sampling.  The maximum legal value is 5.
	If {\tt nojitter} is specified, sample locations
	and times will not be jittered.
\end{defkey}
By default, $3^{2}$ jittered samples are taken.

A given set of sample values must be filtered
in order to
assign a color to a pixel.  Ideally, when performing filtering
for a specific pixel,
the filter will consider samples from neighboring regions.  In
{\rayshade}, the filtering applied to a pixel makes use of samples
taken for that pixel alone.  However, one may increase the size
of the filter that is applied in order to approximate the results
a more robust filtering scheme.

\begin{defkey}{filter}{{\em type} [{\em width}]}
	Use the indicated filter type with the given width,
	in pixels.
	Supported filter types are {\tt gauss} (Gaussian)
	and {\tt box} (the default).
\end{defkey}
The default filter width is 1.0 for a box filter, 1.8 for a Gaussian
filter.  The filter and pixel centers always coincide.
When sampling a pixel, samples are taken over the area of
the pixel filter, which is not necessarily the same as the area
of the pixel itself.

Jittered sampling is used in {\rayshade} to sample extended light
sources as well.  A total of $samples^2$ samples are taken of
each extended light source in order to determine the extent of shadowing.

\section{The Ray Tree}

When ray tracing a scene, reflected or transmitted rays may strike
other reflective or transparent objects.  Further reflected or
transmitted rays will be spawned, and so on.  Taken together, such
a family of rays is termed the {\em ray tree}.  Care must be taken
to control the depth of this tree:  If it is allowed to grow too deeply,
one may spend a great deal of time computing rays that contribute little
to the final picture; if it is not allowed to grow far enough, this
premature tree pruning may be evident in the image.

{\Rayshade} provides two complementary methods for controlling the depth
of the ray tree.  One method sets an absolute maximum for the tree.  The
other allows one to adaptively prune a tree as it grows so that ``unimportant''
rays are not spawned.

\begin{defkey}{maxdepth}{{\em level}}
	Do not spawn rays deeper than those at the given {\em level}.
\end{defkey}
Rays from the eye are of depth zero.  The default value for
{\em level} is 15.
This depth may also be set from the command line through the {\tt -D} option.

\begin{defkey}{cutoff}{{\em threshold}}
	Do not spawn rays whose contribution to the final color of
	the eye ray is less than {\em threshold} for each color channel.
	Threshold may be given as a single floating-point value,
	or as a red-green-blue triple.
\end{defkey}
The default value is 0.002.  This threshold may also be set from
the command line through the {\tt -T} option.
